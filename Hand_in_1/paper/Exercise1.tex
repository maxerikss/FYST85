%\section{First Exercise}
\paragraph{Problem:} Consider the following two-qubit state:
\begin{equation*}
    \ket{\psi} = \frac{1}{2} \ket{00} + \frac{1}{\sqrt{2}} \ket{10} + \frac{1}{2} \ket{11}
\end{equation*}
\begin{enumerate}[A.]
    \item Show that the state is normalized.
    \item If you make a measurement on the first qubit and obtain the result $\ket{0}$, what is now the two-qubit state after this measurement?
    \item Suppose that you instead obtained the result $\ket{1}$, what is now the two-qubit state after this measurement? (Remember to answer as a normalized state)
    \item Is the original state an entangled state?
\end{enumerate}

\paragraph{A.} For a normalized state the coefficients $c_n$ have the property 
\begin{equation*}
    \sum_n \abs{c_n}^2 = 1,
\end{equation*}
thus for the state $\ket{\psi}$ we have the coefficients $c_1 = 1/2$, $c_2 = 1 / \sqrt{2}$, $c_3 = 1 / 2$. Thus
\begin{equation*}
    \sum_n \abs{c_n}^2 = \abs{\frac{1}{2}}^2 + \abs{\frac{1}{\sqrt{2}}}^2 + \abs{\frac{1}{2}}^2 = \frac{1}{4} +\frac{1}{2} + \frac{1}{4} = 1
\end{equation*}
\qed

\paragraph{B.} Let the measurement be described by the set $\{M_0, M_1\}$ where $M_m$ are measurement operators defined as
\begin{equation*}
    M_m = \ket{m}\bra{m} \quad \text{for } m = 0,1
\end{equation*}
where $m$ correspond to the measured value. Since we obtain the result $\ket{0}$ we apply $M_0$ on the first qubit we have 
\begin{align*}
    M_0 \otimes \I (\ket{\psi}) &= M_0 \otimes \I \left( \frac{1}{2} \ket{0} \otimes \ket{0} + \frac{1}{\sqrt{2}} \ket{0} \otimes \ket{1} + \frac{1}{2} \ket{1} \otimes \ket{1} \right)\\
    &= \frac{1}{2} M_0\ket{0} \otimes \I \ket{0} + \frac{1}{\sqrt{2}} M_0 \ket{1} \otimes \I \ket{0} + \frac{1}{2} M_0 \ket{1} \otimes \I \ket{1}\\
    &= \frac{1}{2} \ket{0} \bra{0} \ket{0} \otimes \ket{0} + \frac{1}{\sqrt{2}} \ket{0} \bra{0} \ket{1} \otimes \ket{0} + \frac{1}{2} \ket{0} \bra{0} \ket{1} \otimes \ket{1}\\
    &= \frac{1}{2} \ket{00}.
\end{align*}
Normalizing this by dividing by the square root of the probability of this measurement outcome occurring we get the state after measurement. First we calculate the probability $p(0)$
\begin{align*}
    p(0) &= \bra{\psi} (M_0^\dagger \otimes \I)(M_0 \otimes \I )\ket{\psi}  = \bra{\psi} (M_0 \otimes I) \ket{\psi}\\
    &= \frac{1}{2}\bra{\psi} \ket{00} = \frac{1}{2} \left(  \frac{1}{2} \bra{00}\ket{00} + \frac{1}{\sqrt{2}} \bra{10}\ket{00} + \frac{1}{2}\bra{11}\ket{00} \right) = \frac{1}{4}\bra{00}\ket{00} = \frac{1}{4}
\end{align*}
Thus we get the final state
\begin{equation*}
    \ket{\psi}' = \frac{(M_0 \otimes \I)\ket{\psi}}{\sqrt{p(0)}} = \ket{00}
\end{equation*}

\paragraph{C.} Since we obtain the value $\ket{1}$ we apply $M_1$ to the first qubit. We get
\begin{align*}
    M_1 \otimes \I (\ket{\psi}) &= M_1 \otimes \I \left( \frac{1}{2} \ket{0} \otimes \ket{0} + \frac{1}{\sqrt{2}} \ket{0} \otimes \ket{1} + \frac{1}{2} \ket{1} \otimes \ket{1} \right)\\
    &= \frac{1}{2} M_0\ket{0} \otimes \I \ket{0} + \frac{1}{\sqrt{2}} M_0 \ket{1} \otimes \I \ket{0} + \frac{1}{2} M_0 \ket{1} \otimes \I \ket{1}\\
    &= \frac{1}{2} \ket{1} \bra{1} \ket{0} \otimes \ket{0} + \frac{1}{\sqrt{2}} \ket{1} \bra{1} \ket{1} \otimes \ket{0} + \frac{1}{2} \ket{1} \bra{1} \ket{1} \otimes \ket{1}\\
    &= \frac{1}{\sqrt{2}} \ket{10} + \frac{1}{2} \ket{11}.
\end{align*}
Since we must have $\sum_m p(m) = 1$ we get from \textbf{B.} that $p(1) = 3/4$. Thus we get the final state
\begin{equation*}
    \ket{\psi}' = \frac{(M_1 \otimes \I) \ket{\psi}}{\sqrt{p(1)}} = \frac{2}{\sqrt{3}} (\frac{1}{\sqrt{2}} \ket{10} + \frac{1}{2} \ket{11}) = \frac{\sqrt{2}}{\sqrt{3}} \ket{10} + \frac{1}{\sqrt{3}} \ket{11}
\end{equation*}

\paragraph{D.} If the state is separable and not entangled we can write it as $\ket{A} \otimes \ket{B}$ where 
\begin{align*}
    \ket{A} &= a_0 \ket{0} + a_1 \ket{1}\\
    \ket{B} &= b_0 \ket{0} + b_1 \ket{1}.
\end{align*}
Performing the product we obtain
\begin{align*}
    \ket{A} \otimes \ket{B} &= (a_0 \ket{0} + a_1 \ket{1}) \otimes (b_0 \ket{0} + b_1 \ket{1}) \\
    &= a_0 b_0 \ket{00} + a_0 b_1 \ket{01} + a_1 b_0 \ket{01} + a_1 b_1 \ket{11}.
\end{align*}
So if the state is separable there is a unique solution to the system
\begin{equation*}
    \left\{\begin{array}{@{}>{\displaystyle}l@{}}
        a_0 b_0 = \frac{1}{2}\\
        a_0 b_1 = 0\\
        a_1 b_0 = \frac{1}{\sqrt{2}}\\
        a_1 b_1 = \frac{1}{2} 
    \end{array}\right..
\end{equation*}
For the second equation we must have either $a_0 = 0$ or $b_1 = 0$. However, $a_0 = 0$ contradicts the first equation and $b_1 = 0$ contradicts the last equation. Thus $\ket{\psi}$ is entangled.