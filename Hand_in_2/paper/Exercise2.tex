\section*{Exercise 2.72}
\paragraph{Problem:} \textbf{(Bloch sphere for mixed states)} The Bloch sphere picture for pure states of a single qubit was introduced in Section 1.2. This description has an important generalization to mixed states as follows.
\begin{enumerate}[label=(\arabic*)]
    \item Show that an arbitrary density matrix for a mixed state qubit may be written as 
        \begin{equation}
            \rho = \frac{\I + \vec{r} \cdot \vec{\sigma}}{2},
        \end{equation}
    where $\vec{r}$ is ta real three-dimensional vector such that $\norm{\vec{r}} \leq 1$. This vector is known as the \textit{Bloch vector} for the state $\rho$.
    \item What is the Bloch vector representation for the state $\rho = \I / 2$?
    \item Show that a state $\rho$ is pure if and only if $\norm{\vec{r}} = 1$.
    \item Show that for pure states the description of the Bloch vectors we have given coincides with that in Section 1.2.
\end{enumerate}

\paragraph{(1) Solution:} We can start by creating a general density matrix for a qubit, which will be in the form
\begin{equation}
    \rho = \begin{bmatrix}
        c_{00} & c_{01}\\
        c_{10} & c_{11}
    \end{bmatrix},
\end{equation}
where $c_{ij} \in \mathbb{C}$ for $i,j = 0,1$. Then we have three conditions on a density matrix
\begin{enumerate}[label=\textit{\roman*}., leftmargin=2cm]\singlespacing
    \item $\tr\rho = 1$
    \item $\rho$ is unitary
    \item $\rho$ is positive
\end{enumerate}
The first condition gives that $c_{00} + c_{11} = 1$.Unitarity gives that $c_{00} = c_{00}^*$, $c_{11} = c_{11}^*$, $c_{01} = c_{10}^*$. This gives us a possible parametrization 
\begin{equation}
        c_{00} = \frac{1+z}{2}, \quad c_{11} = \frac{1-z}{2}, \quad c_{01} = \frac{x - iy}{2}, \quad c_{10} = \frac{x+iy}{2}.
\end{equation}
Rewriting $\rho$
\begin{equation}
    \rho = \frac{1}{2}
    \begin{bmatrix}
        1+z & x - iy\\
        x + iy & 1- z    
    \end{bmatrix}.\label{eq:param}
\end{equation}
Now we can separate this matrix such that
\begin{equation}
    \begin{bmatrix}
        1+z & x+ iy\\
        x - iy & 1- z    
    \end{bmatrix}
    =
    \begin{bmatrix}
        1 & 0\\
        0 & 1
    \end{bmatrix}
    +z
    \begin{bmatrix}
        1 & 0\\
        0 & -1
    \end{bmatrix}
    +x
    \begin{bmatrix}
        0 & 1\\
        1 & 0
    \end{bmatrix}
    +y
    \begin{bmatrix}
        0 & -i\\
        i & 0
    \end{bmatrix}.
\end{equation}
We can identify these as the Pauli matrices and thus
\begin{equation}
    \rho = \frac{1}{2} (\I + x\sigma_x + y \sigma_y + z\sigma_z) = \frac{\I + \vec{r}\cdot \vec{\sigma}}{2},
\end{equation}
where $\vec{r} = (x,y,z)$ and $\vec{\sigma} =(\sigma_x,\sigma_y,\sigma_z)$. To verify $\norm{\vec{r}} \leq 1$ we use the positivity condition which is equivalent to say that all eigenvalues $\lambda$ of $\rho$ must be positive. Thus, using Eq. \eqref{eq:param} to calculate the eigenvalues we obtain
{\everymath{\displaystyle}
\begin{align}
    0 &= \det (\rho - \lambda \I) = 
    \begin{vmatrix}
        \frac{1 + z}{2} - \lambda & \frac{x - iy}{2}\\[8pt]
        \frac{x + iy}{2} & \frac{1- z}{2} - \lambda
    \end{vmatrix} \\
    &= \left(\frac{1 + z}{2} - \lambda\right) \left(\frac{1- z}{2} - \lambda\right) - \frac{x -iy}{2} \cdot \frac{x + iy}{2}\\
    &= \frac{1- z^2}{4} -\left( \frac{1 + z}{2} + \frac{1-z}{2} \right)\lambda + \lambda^2 - \frac{x^2 + y^2}{4}\\
    &= \lambda^2 - \lambda - \frac{x^2 + y^2 + z^2 - 1}{4}.
\end{align}}
Using the $pq$-formula with
\begin{equation}
    p = -1, \quad q = - \frac{x^2 + y^2 + z^2 - 1}{4}
\end{equation}
we get
\begin{equation}
    \lambda = -\frac{p}{2} \pm \sqrt{\frac{p^2}{4} - q} = \frac{1}{2} \pm \sqrt{\frac{x^2 + y^2 + z^2}{4}} = \frac{1}{2} \left(1 \pm \sqrt{\norm{\vec{r}}^2} \right).\label{eq:pq}
\end{equation}
From the definition of a density matrix we know that for the spectral decomposition the eigenvalues are probabilities. Thus $0 \leq \lambda \leq 1$, and it then follows that $\norm{\vec{r}} \leq 1$.\\
\qed

\paragraph{(2) Solution:} For the density matrix to equal $\rho = \I / 2$ we must have that $\vec{r} \cdot \vec{\sigma} = \vec{0}$. Since $\vec{\sigma}$ is non-zero we must have that $\vec{r} = \vec{0}$.\\\qed

\paragraph{(3) Solution:} The density matrix of a pure state $\ket{\psi}$ can be written $\rho = \ket{\psi} \bra{\psi}$. That is, the probability of obtaining this state is 1. So if we do a spectral decomposition of the density matrix the eigenvalues must be $\lambda = 0,1$. And by Eq. \eqref{eq:pq} this happens if and only if $\norm{\vec{r}} = 1$

\paragraph{(4) Solution:} According to section 1.2 we can write a pure state as 
\begin{equation}
    \ket{\psi} = \cos\frac{\theta}{2}\ket{0} + e^{i\varphi}\sin\frac{\theta}{2}\ket{0}.
\end{equation}
Since all pure states lie on the Bloch sphere, which has a radius of 1. We can switch to spherical coordinates with
\begin{equation}
    x = \sin\theta \cos\varphi, \quad y = \sin\theta \sin\varphi, \quad z = \cos\theta.
\end{equation}
Rewriting $\rho$ in Eq. \eqref{eq:param} we obtain
{\everymath{\displaystyle}
\begin{align}
    \rho &= \frac{1}{2} 
    \begin{bmatrix}
        1 + \cos\theta & \sin\theta  \cos\varphi - i \sin\theta \sin\varphi\\
        \sin\theta  \cos\varphi + i \sin\theta \sin\varphi & 1 - \cos\theta
    \end{bmatrix}\\
    &= 
    \begin{bmatrix}
        \frac{1 + \cos\theta}{2} & (\cos\varphi - i\sin\varphi)\frac{\sin\theta}{2}\\[8pt]
    (\cos\varphi + i\sin\varphi)\frac{\sin\theta}{2} &  \frac{1 - \cos\theta}{2}
    \end{bmatrix}
\end{align}
Using Euler's formula we can rewrite $\cos\varphi \pm i\sin\varphi = e^{\pm i \varphi}$. Then using the trigonometric identities $2\cos^2\theta = 1 + \cos 2\theta$, $2 \sin^2\theta = 1 - \cos 2\theta$ and $\sin 2\theta = 2\sin\theta \cos\theta$, we rewrite the matrix as
\begin{align}
    \rho &= 
    \begin{bmatrix}
        \cos^2\frac{\theta}{2} & e^{-i\varphi}\sin\frac{\theta}{2}\cos\frac{\theta}{2}\\[8pt]
        e^{i\varphi}\sin\frac{\theta}{2}\cos\frac{\theta}{2} & \sin^2\frac{\theta}{2}
    \end{bmatrix}.
\end{align}
Changing to state vector representation we get
\begin{align}
    \rho &\;\dot{=} \cos^2\frac{\theta}{2} \ket{0}\bra{0} + e^{-i\varphi}\sin\frac{\theta}{2}\cos\frac{\theta}{2} \ket{0}\bra{1} + e^{i\varphi}\sin\frac{\theta}{2}\cos\frac{\theta}{2}\ket{1}\bra{0} + \sin^2\frac{\theta}{2} \ket{1}\bra{1}\\
    &= \left(\cos\frac{\theta}{2} \ket{0} + e^{i\varphi}\sin\frac{\theta}{2}\ket{1}\right) \cdot \left(\cos\frac{\theta}{2} \bra{0} + e^{-i\varphi}\sin\frac{\theta}{2}\bra{1}\right) \\
    &= \ket{\psi} \bra{\psi}.
\end{align}}
\qed
